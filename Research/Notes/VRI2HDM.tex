\documentclass[a4paper,10pt]{article}

\usepackage[utf8x]{inputenc}
\usepackage[textwidth=170mm,textheight=229mm]{geometry}
\usepackage{graphicx}
\usepackage{amsmath}
\usepackage{fancybox}
\usepackage{cancel}
\usepackage{wrapfig}
\usepackage{amsmath,amssymb,amsfonts,mathrsfs}
\usepackage{color}
\usepackage{slashed}
\usepackage{bbm}  
\usepackage{xspace}
\usepackage{subcaption}
\usepackage{multicol}



\usepackage{tikz}
\usetikzlibrary{arrows,shapes}
\usetikzlibrary{trees}
\usetikzlibrary{matrix,arrows} 				% For commutative diagram
											% http://www.felixl.de/commu.pdf
\usetikzlibrary{positioning}				% For "above of=" commands
\usetikzlibrary{calc,through}				% For coordinates
\usetikzlibrary{decorations.pathreplacing}  % For curly braces
% http://www.math.ucla.edu/~getreuer/tikz.html
\usepackage{pgffor}							% For repeating patterns

\usetikzlibrary{decorations.pathmorphing}	% For Feynman Diagrams
\usetikzlibrary{decorations.markings}
\tikzset{
	% >=stealth', %%  Uncomment for more conventional arrows
    vector/.style={decorate, decoration={snake}, draw},
	provector/.style={decorate, decoration={snake,amplitude=2.5pt}, draw},
	antivector/.style={decorate, decoration={snake,amplitude=-2.5pt}, draw},
    fermion/.style={draw=black, postaction={decorate},
        decoration={markings,mark=at position .55 with {\arrow[draw=black]{>}}}},
    fermionbar/.style={draw=black, postaction={decorate},
        decoration={markings,mark=at position .55 with {\arrow[draw=black]{<}}}},
    fermionnoarrow/.style={draw=black},
    gluon/.style={decorate, draw=black,
        decoration={coil,amplitude=4pt, segment length=5pt}},
    scalar/.style={dashed,draw=black, postaction={decorate},
        decoration={markings,mark=at position .55 with {\arrow[draw=black]{>}}}},
    scalarbar/.style={dashed,draw=black, postaction={decorate},
        decoration={markings,mark=at position .55 with {\arrow[draw=black]{<}}}},
    scalarnoarrow/.style={dashed,draw=black},
    electron/.style={draw=black, postaction={decorate},
        decoration={markings,mark=at position .55 with {\arrow[draw=black]{>}}}},
	bigvector/.style={decorate, decoration={snake,amplitude=4pt}, draw},
}\usetikzlibrary{decorations.markings}

% TIKZ - for block diagrams, 
% from http://www.texample.net/tikz/examples/control-system-principles/
% \usetikzlibrary{shapes,arrows}
\tikzstyle{block} = [draw, rectangle, 
    minimum height=3em, minimum width=6em]

%Definiciones
\newcommand{\phidag}{\phi^{\dagger}}
\newcommand{\vdag[1]}{V^{\dagger}_{#1}}
\newcommand{\Vdag[1]}{V^{#1\dagger}}
\newcommand{\Lbar}{\overline{L}}
\newcommand{\PL}{\frac{(1+\gamma_5)}{2}}
\newcommand{\PR}{\frac{(1-\gamma_5)}{2}}
\newcommand{\ld[1]}{\lambda_{#1}}
\newcommand{\Mv[1]}{M_{V_{#1}}}
\newcommand{\Mvc}{M_{V^{\pm}}}
\newcommand{\Tr}{\text{Tr}}
\newcommand{\pmu}{\partial_{\mu}}
\newcommand{\pnu}{\partial_{\nu}}
\newcommand{\bpsi}{\overline{\psi}}
\newcommand{\SM}{\textbf{SM}}
\newcommand{\CMB}{\textbf{CMB}}
\newcommand{\pbar[1]}{\overline{#1}}
\newcommand{\Deriv}{\cancel{D}}
\newcommand{\OX}{\overline{X}}
\newcommand{\A[1]}{A_{\mu #1}}
\newcommand{\AF[1]}{A_{\mu\nu #1}}
\newcommand{\AFD[1]}{A^{\mu\nu #1}}
\newcommand{\U[1]}{U_{#1}}
\newcommand{\UD[1]}{U_{#1}^{\dagger}}

\begin{document}
\section{Modelo}
Consideramos un modelo quiral tipo $SU(2)_1\otimes SU(2)_2 \otimes U(1)$ que, a través de un mecanismo sigma no lineal, se rompe al grupo de simetría electrodébil del modelo estandar $SU(2)_L \otimes U(1)_Y$. Si miramos solamente el grupo quiral el siguiente contenido de campos: 2 campos de gauge $\A[1]$ y $\A[2]$ asociados a cada grupo de simetría, 3 campos escalares $\phi_1$, $\phi_2$ y $\Sigma$. Los campos de gauge transforman de la siguiente manera
\begin{eqnarray*}
\A[1] &\rightarrow& \U[1]\A[1]\UD[1] - \frac{i}{g_1}(\partial \U[1])\UD[1] \\
\A[2] &\rightarrow& \U[2]\A[2]\UD[2] - \frac{i}{g_2}(\partial \U[2])\UD[2]
\end{eqnarray*}
donde $g_1$ y $g_2$ son las constantes de acoplamiento de cada grupo respectivamente. $\U[1]$ y $\U[2]$ son matrices de transformacion definidas como
\begin{equation}
\U[1] = e^{i\vec{\beta}_1\cdot\vec{\tau}/2} \qquad \U[2] = e^{i\vec{\beta}_2\cdot\vec{\tau}/2}
\end{equation} 
$\vec{\tau}$ son las matrices de pauli. La transformación de los campos escalares está definida de la siguiente manera
\begin{eqnarray*}
SU(2)_1: &&\quad \phi_1 \rightarrow \U[1]\phi_1 \qquad \phi_2 \rightarrow \phi_2 \qquad \phi_1 \sim (2,1) \\
SU(2)_2: &&\quad \phi_1 \rightarrow \phi_1 \qquad \phi_2 \rightarrow \U[2]\phi_2 \qquad \phi_2 \sim (1,2) 
\end{eqnarray*}
\begin{equation*}
\Sigma \rightarrow \U[1]\Sigma \UD[2]
\end{equation*}
En base a esto podemos escribir el lagrangiano que involucra al sector de gauge más el sector escalar
\begin{eqnarray}
\nonumber \mathscr{L} &=& -\frac{1}{4}\AF[1]\AFD[1] - \frac{1}{4}\AF[2]\AFD[2] - (D_{\mu}\phi_1)^{\dagger}(D^{\mu}\phi_1) - (D_{\mu}\phi_2)^{\dagger}(D^{\mu}\phi_2) + \\ 
&& v^2\Tr\left[D_{\mu}\Sigma)^{\dagger}(D^{\mu}\Sigma)\right] + V(\phi_1,\phi_2,\Sigma)
\end{eqnarray}
Podemos definir las derivadas covariantes como
\begin{eqnarray}
\nonumber D_{\mu}\Sigma &=& \partial_{\mu}\Sigma - ig_1\A[1]\Sigma + ig_2\Sigma\A[2] \\
D_{\mu}\phi_1 &=& \partial_{\mu}\phi_1 + ig_1\A[1]\phi + ig_yB_{\mu}\phi_1 \\
\nonumber D_{\mu}\phi_2 &=& \partial_{\mu}\phi_2 + ig_2\A[1]\phi + ig_yB_{\mu}\phi_2 \label{deriv}
\end{eqnarray}
Al romper la simetría nos paramos en la gauge unitario $\langle \Sigma \rangle = 1$ el término cinético del campo escalar $\Sigma$ queda como
\begin{eqnarray}
u^2\Tr\left[D_{\mu}\Sigma)^{\dagger}(D^{\mu}\Sigma)\right] = u^2\Tr\left[g_1\A[1]-g_2\A[2]\right]^2
\end{eqnarray}
Luego de la ruptura de la primera simetría el lagrangiano queda como
\begin{eqnarray}
\nonumber \mathscr{L} &=& -\frac{1}{4}\AF[1]\AFD[1] - \frac{1}{4}\AF[2]\AFD[2] -(D_{\mu}\phi_1)^{\dagger}(D^{\mu}\phi_1) -(D_{\mu}\phi_2)^{\dagger}(D^{\mu}\phi_2) + \\ 
&& u^2\Tr\left[g_1\A[1]-g_2\A[2]\right]^2 + V(\phi_1,\phi_2)
\end{eqnarray}
En este nuevo lagrangiano seguimos teniendo 2 campos de gauge que transforman ahora bajo el grupo de simetría diagonal $SU(2)_L$ de la sifuiente forma
\begin{eqnarray*}
\A[i] &\rightarrow& U\A[i]U^{\dagger} - \frac{i}{g_i}(\partial U)U^{\dagger}
\end{eqnarray*}
Donde $U = e^{i\vec{\beta}_L\cdot\vec{\tau}/2}$ con $\vec{\beta}_L = \vec{\beta}_1 = \vec{\beta}_2$. Sin embargo ahora aparece un término explícito de masa que involucra a ambos campos. En este punto podemos realizar la ruptura de simetría del grupo $SU(2)_L\times U(1)_Y \rightarrow U(1)_{em}$ considerando que solo el campo escalar $\phi_1$ es el que adquiere un valor de espectación del vacío $\langle \phi_1 \rangle = v$. Trabajando en la base $(\A[1]^3,\A[2]^3,B_{\mu})$, la matriz de masa para el sector neutro queda expresada como
\begin{eqnarray}
M_{\text{neutra}} = \frac{v^2}{4}
\begin{bmatrix}
 (1+a^2)g_1^2  & -a^2g_1g_2  & -g_1g_y \\
-a^2g_1g_2     & a^2g_2^2    & 0  \\
-g_1g_y        & 0           & g_y^{2} \\
\end{bmatrix}
\end{eqnarray}
Considerando la base $(\A[1]^{\pm},\A[2]^{\pm})$, la matriz de masa para el sector cargado queda expresada como
\begin{eqnarray}
M_{\text{cargada}} = \frac{v^2}{4}
\begin{bmatrix}
(1+a^2)g_1^2  & -g_1g_2a^2   \\
-g_1g_2a^2    & a^2g_2^2
\end{bmatrix}
\end{eqnarray}
donde $a=u/v$. La matriz de masa puede ser diagonalizada en el límite donde $g_2 \gg g_1$ y manteniendo términos de orden $g_1/g_2$. Con esta aproximación obtenemos las siguientes mezclas (Autoestados de masa en función de autoestados de sabor)
\begin{eqnarray}
\nonumber A &=& \frac {g_y}{\sqrt{g_1^2+g_y^2}}\A[1]^3 + \frac{g_1g_y}{g_2\sqrt{g_1^2+g_y^2}}\A[2]^3 + \frac{g_1}{\sqrt{g_1^2+g_y^2}}B_{\mu}\\[10pt]
\nonumber Z &=& -\frac{g_1}{\sqrt{g_1^2+g_y^2}}\A[1]^3 - \frac{g_1^2}{g_2\sqrt{g_1^2+g_y^2}}\A[2]^3 + \frac{g_y}{\sqrt{g_1^2+g_y^2}}B_{\mu}  \\[10pt]
\nonumber \rho^0 &=& -\frac{g_1}{g_2}\A[1]^3 + \A[2]^3 \\[10pt]
\nonumber W^{\pm} &=& \A[1]^{\pm} + \frac{g_1}{g_2}\A[2]^{\pm} \\[10pt]
\rho^{\pm} &=& - \frac{g_1}{g_2}\A[1]^{\pm} + \A[2]^{\pm} 
\end{eqnarray}
donde
\begin{eqnarray}
\A[1]^{\pm} &=& \frac{1}{\sqrt{2}}(\A[1]^1 \mp i\A[1]^2) \\
\A[2]^{\pm} &=& \frac{1}{\sqrt{2}}(\A[2]^1 \mp i\A[2]^2)
\end{eqnarray}


\end{document}

