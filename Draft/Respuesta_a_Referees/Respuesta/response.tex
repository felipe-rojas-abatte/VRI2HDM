\documentclass[10pt,a4paper]{letter}
\usepackage[latin1]{inputenc}
\usepackage{amsmath}
\usepackage{amsfonts}
\usepackage{amssymb}
\address{Alfonso R. Zerwekh\\Email:alfonso.zerwekh@usm.cl} 
\signature{The authors of manuscript DG12224} 
\begin{document} 
	\begin{letter}{Joshua C. Sayre\\ 
			Associate Editor\\
			Physical Review D\\
			Email: prd@aps.org} 
		\opening{Dear Editor,} 
		
		We are very thankful to the referees because of their favorable and constructive comments. In the following paragraphs we will provide answers to their questions.
		
		{\bf Response to Referee A}
		\\
		\\
		{\bf Referee:}
\begin{quotation}
	1. One is the problem with perturbative unitarity in the presence of a
	large gauge coupling $g_2$. Although other unitarity conditions on Higgs
	quartic couplings are imposed, there is no discussion on the potential
	violation of unitarity due to new resonances or strongly coupled dark
	Higgs near the mass of rho gauge bosons.
\end{quotation}	

{\bf Answer:}\\
We agree with the referee in the sense that the introduction of massive vector resonance without a companion scalar degree of freedom (a Higgs boson) would, in principle, induce a dangerous unitarity violation. Indeed this a common aspect of effective theories, as ours, and what it must be shown is that the scale of unitarity violation is beyond the expected range of validity of the model. In our case, for instance, the $W^{+}W^{-}\rho^0$ vertex is proportional to $g_1 \cos(\theta_W)({g_1^3 \over g_2^3}) ({v^2 \over u^2})$ which is very small and beyond the level of approximation we worked at. However, we computed the contribution of the $\rho^0$ exchange to the process $W_L W_L \rightarrow W_L W_L$. We found that the corresponding scale of unitarity violation is:

\[
\Lambda_{\mathrm{uv}}\approx \left(\frac{2\sqrt{6\pi}M_{\rho}M_Z}{M_W u}\right)^{1/2}\frac{M_{\rho}}{M_W} v
\]

In our case, the lowest value of $\Lambda_{\mathrm{uv}}$ is obtained for $u=5v\approx 1.2$ TeV and $M_{\rho}=2$ TeV, thus we found that $\Lambda_{\mathrm{uv}} > 25$ TeV. On the other hand, we expect that the maximum scale at which our model remains valid is $4\pi u \in [9,15]$ TeV.
\\
\\
\newpage
{\bf Changes in the text:}\\
Consequently, in section III, at the end of the item called "Unitarity", we included a paragraph reflecting the explanation given above.
\\
\\
{\bf Referee:}
\begin{quotation}
	2. Another is the bound from electroweak precision data (EWPD) on the
	mixing between the electroweak gauge bosons and the extra gauge
	bosons. Although the authors assumed that the mixing is small for a
	small $g_1/g_2$, there was no explicit discussion on EWPD bounds.
\end{quotation}

{\bf Answer:}\\
Concerning the EWPD, we computed the tree level correction to the $T$ parameter. Our result is:

\[
\Delta T_{\mathrm{tree}}\approx\frac{{{\tan^{2}(\theta_W)}}\, {{\mathit{M_W}}^{6}}\, {{a}^{2}}}{{\alpha_{\mathrm{EM}}{\mathit{M_{\rho} }}^{6}}}
\]

For the values of $a$ and $M_{\rho}$ considered in our manuscript, $\Delta T_{\mathrm{tree}}$ turns out to be very small: $\Delta T_{\mathrm{tree}}\sim 10^{-9}-10^{-5} $ which is in perfect agreement with current limits on  $\Delta T$ at 95\% C.L. Additionally, in models with very massive vector resonances it is expected that the $S$ parameter be proportional to $T$ (Phys. Rev. D71, 035007) and, consequently, we expect $\Delta S_{\mathrm{tree}}$ to be also small. In what respect to loop contributions, our experience which models containing heavy vector resonances (Eur.Phys.J. C74 (2014) 2822,  arXiv:1707.05195) shows that for small mixing angle and large resonance mass (as is our case), the one-loop contribution are consistent with experimental limits.

{\bf Changes in the text:}\\
Consequently, in section III, at the end of the item called "Electroweak precision Test", we included a paragraph reflecting the explanation given above.
\newpage
{\bf Response to Referee B}
\\
\\
	{\bf Referee:}
\begin{quotation}
	after equation 2 on page 3: it is not completely true that the $Z_2$
	symmetry prevents $\phi_2$ from getting a vev. With the appropriate value
	for $m_2^2$ (if condition 13 is violated for example), $\phi_2$ would
	acquire a vev, spontaneously breaking the $Z_2$.
\end{quotation}	

{\bf Answer:}\\
The Referee is absolutely right. We changed the sentence to: ``Notice that we assume that $\phi_{2}$ does not acquires a v.e.v."
\\
\\
	{\bf Referee:}
\begin{quotation}
	The caption of figure 2 should include an explanation for the dotted
	line.
\end{quotation}	

{\bf Answer:}\\
We fixed the caption
\\
\\
{\bf Referee:}
\begin{quotation}
	in section V, the authors argue that the cross-section for mono-Z at
	LHC in their model is increased with respect to a typical inert
	doublet. But there is no comment on the observability of such
	cross-section. If possible such a comment would add to the discussion.
\end{quotation}

{\bf Answer:}\\
We used the package CheckMATE to compare the predictions of our model to available experimental data. Unfortunately we had access only to LHC data at $\sqrt{s}=8$ TeV and a luminosity of $20.3$ fb$^{-1}$. However we extrapolate the result for the LHC at $\sqrt{s}=13$ TeV and a luminosity of $3000$ fb$^{-1}$ assuming that the significance of the signal with respect to the background remains the same. Consequently we add, at the end of section V a discussion on this analysis and we a two new plots (figure 10)  showing the results.


I this way, we think we have addressed all the questions and comments from the Referees.




	
		\closing{Sincerely yours} 
		%\cc{Cclist} 
		%\ps{adding a postscript} 
		%\encl{list of enclosed material} 
	\end{letter} 
\end{document}